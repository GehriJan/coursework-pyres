\documentclass{scrartcl}
\usepackage{fontenc}

\title{Dispositionspapier zur Studienarbeit\\Titel der Arbeit}
\author{ Ihr Name }
\date{xx.01.2025}

\begin{document}

\maketitle

\section{Kurzbeschreibung der Arbeit}
<Worum geht es in der Arbeit? Wie ist die (aktuelle) Ausgangssituation? Welches
Themenfeld wird bearbeitet? Welche Problemstellung soll angegangen werden? Welche
Grundlagen müssen vorhanden sein und welche Randbedingungen sind gegeben? Welche
Zielsetzungen gibt es in dieser Arbeit? Welche methodische Vorgehensweise wird
gewählt?>

\emph{Dies soll möglichst in einem Fließtext dokumentiert werden. Idealerweise
abschließend mit sehr konkreten Zielbeschreibungen, die auch validierbar sind\/}.


\section{Gliederung und Zeitplan}
<Identifikation der wesentlichen Arbeitsschritte. Meilensteinplan. Konsequenzen
und Möglichkeiten der Meilensteine. Zeitplan bis zur Beendigung des praktischen
Teils sowie der Dokumentation.> \emph{Das gehört nicht in die endgültige Ausarbeitung!\/}

<Eine erste Gliederung der Arbeit. Benennung von
Kapiteln und Unterkapiteln. Dies gilt als Leitfaden und ist noch nicht als abschließend.>

\section{Grundlegende Literatur}
<Belegen der Ausgangssituation. Wer hat auf ähnlichem Themenfeld bereits
gearbeitet? Wie passt die Studienarbeit in die aktuelle wissenschaftliche
Landschaft und was ist neu (dies wird oben dargelegt und hier belegt). Was wird
durch die erstellte Lösung verbessert und wie wird dies nachgewiesen?>

\end{document}
